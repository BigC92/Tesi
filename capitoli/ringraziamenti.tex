Eccomi qui alla conclusione di un lungo viaggio iniziato 4 anni fa ormai. In questo lungo periodo ho incontrato tanta gente, ma soprattutto trovato delle belle amicizie. Cercherò in questo breve testo di ringraziarvi tutti come è giusto che sia, nel caso mi scordassi di qualcuno non prendetevela la memoria, come molti di voi sanno non è il mio miglior pregio.

Prima fra tutti Marta, la mia ragazza. Non basterebbe un libro per ringraziarti del supporto e degli stupendi momenti trascorsi in questi 4 anni, grazie di esserci stata in passato e di esserci ora, grazie per l'aiuto nelle sessioni d'esame, nel supporto morale che mi hai dato mentre portavo avanti questo tirocinio e nella stesura della mia tesi. Lo sappiamo entrambi che presto ricambierò, è quasi giunto anche il tuo di momento.

Grazie ai miei genitori e alla mia famiglia per aver sempre creduto in me e per avermi supportato nel corso di tutti i miei studi, questo traguardo per me conta moltissimo e spero presto di potermi "sdebitare" per tutti gli sforzi che avete fatto per renderlo possibile.

Un ringraziamento speciale voglio farlo a tutti i miei compagni di viaggio che hanno condiviso con me fantastici momenti, in un perfetto mix tra studio e divertimento: Angelo, Armando e Davide per le serate trascorse insieme fuori dall'Università, per le risate più assurde nei momenti più improbabili, sempre per merito di qualche cavolata combinata dal pagliaccio del gruppo, Armando ovviamente! 

Grazie a Dario, per i sabati sera trascorsi tra una birra a San Lorenzo e una risata per ogni cretinata ci passasse per la testa, per il supporto morale che mi hai dato nei momenti più bui trascorsi in questi anni. Ancora ricordo che cosa eravamo in grado di creare quando preparavamo insieme gli esami, ma ci sono persone che potrebbero leggere con troppo scetticismo, evitiamo di lasciare tracce scritte!

Grazie ai compagni, mattinieri, della DS2: Flavia e Matteo. Aspettare con voi l'inizio di deprimenti giornate di lezioni e studio è stato sempre un piacere e mi ha aiutato ad affrontarle, almeno, col sorriso.

Grazie a tutte le persone con cui ho iniziato questo percorso: Francesca, Giuseppe, Gregorio (i tuoi PDF sono stati e saranno sempre una salvezza), Pierluigi, Luca, Kevin, Claudio, Emanuele, Michele e Leonardo. Spero di non aver scordato nessuno, ma consideratevi comunque tutti nell'elenco!

Grazie a tutta Roma Tre Radio, ma soprattutto a Vincenzo, Graziana e Stefano. Affrontare con voi i pomeriggi del venerdì tra ospitate e notizie dal mondo della musica non è stato solo un piacere, è stato molto di più. Entrare in radio il venerdì pomeriggio è come tornare a casa. Non mi resta che dirvi: MU! 

Grazie poi a tutte le persone che ho conosciuto e incontrato al di fuori delle mura universitarie, ma è inutile che vi elenchi tutte, siete troppi!\\Spero di non aver saltato nessuno, ma se così non fosse grazie comunque a tutti!\\\\E che Ronnie James vi guidi tutti!