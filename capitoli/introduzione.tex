Molto spesso utilizziamo le mappe o ci imbattiamo nell'annoso problema di voler visualizzare un'area ben definita di una mappa su un browser web, il problema che ci troviamo davanti è che l'area visualizzata è ridotta o non corretta. 

Sono diversi i servizi di visualizzazione di mappe presenti in rete, ma in nessuno di essi è possibile visualizzare la mappa con una vista prospettica. Più in particolare in nessuno è possibile visualizzare la mappa \textit{stradale} con una prospettiva di 45 gradi; Google Maps permette di utilizzare le sue mappe \textit{satellitari} con un'inclinazione di 45 gradi, andando oltre la semplice visualizzazione della mappa, mostrando la struttura tridimensionale della terra e degli edifici, questo, però, risulta spesso fuorviante o, in alcuni casi, anche troppo pesante per il proprio computer, che fatica a caricare la pagina o semplicemente a renderizzare correttamente la mappa.

Proprio per questo lo scopo del tirocinio è stato quello di creare un sistema che al pari dei sistemi di visualizzazione già presenti nel web, permettesse di interagire con la mappa, navigando tra le diverse aree geografiche e permettendo anche di effettuare le azioni di \textit{zoom in} o \textit{zoom out} per aumentare o diminuire i dettagli delle informazioni visualizzate o semplicemente la grandezza dell'area visualizzata.

In breve verrà ora riassunto il contenuto dei capitoli che seguiranno:
\begin{itemize}
	\item \textbf{Capitolo 1}: in questo capitolo verranno analizzati e descritti i sistemi di visualizzazione di mappe più comuni e conosciuti da sviluppatori e utenti comuni, facendo anche riferimento alle banche dati, se necessario, da cui questi servizi ottengono le mappe che poi saranno mostrate sul browser web;
	\item \textbf{Capitolo 2}: in questo capitolo sono descritte le tecnologie e le API di riferimento che sono state utilizzate per la realizzazione del progetto;
	\item \textbf{Capitolo 3}: questo è la sezione in cui vengono spiegati gli obbiettivi e i requisiti del progetto;
	\item \textbf{Capitolo 4}: è il capitolo centrale del progetto, in cui vengono descritte nel dettaglio tutte le componenti: il server sviluppato in node.js, il client sviluppato utilizzando la libreria Three.js e gli algoritmi o i modelli matematici che hanno permesso la realizzazione del progetto;
	\item \textbf{Capitolo 5}: in questa parte sarà possibile vedere le immagini e il funzionamento, per quanto possibile, del progetto;
	\item \textbf{Conclusioni}: in questa parte vengono infine descritti gli scopi raggiunti e quali potrebbero essere gli sviluppi futuri del progetto.
\end{itemize}