Molto spesso utilizziamo le mappe o ci imbattiamo nell'annoso problema di voler visualizzare un'area ben definita di una mappa su un browser web, il problema che ci troviamo davanti \'e che l'area visualizzata \'e ridotta o non corretta. 


Sono diversi i servizi di visualizzazione di mappe presenti in rete, ma in nessuno di essi \'e possibile visualizzare la mappa con una vista prospettica. Pi\'u in particolare in nessuno \'e possibile visualizzare la mappa \textit{stradale} con una prospettiva di 45 gradi; Google Maps permette di utilizzare le sue mappe \textit{satellitari} con un'inclinazione variabile tra i 15 e i 45 gradi, andando oltre la semplice visualizzazione della mappa, mostrando la struttura tridimensionale della terra e degli edifici, questo, per\'o, risulta spesso fuorviante o, in alcuni casi, anche troppo pesante per il proprio computer.